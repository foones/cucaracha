\documentclass{article}
\usepackage[spanish]{babel}
\usepackage[utf8]{inputenc}
\usepackage{amssymb}
\usepackage{xcolor}
\usepackage{xspace}
\usepackage{listings}
\usepackage{tabularx}
\usepackage[margin=1.9cm]{geometry}
\usepackage{amsmath}
\usepackage{listings}

\lstset{
  escapeinside={(*@}{@*)},
  basicstyle=\tt,
  literate={.entrada}{\em Entrada:}8
           {.salida}{\em Salida:}7
           {.if}{\bf if}2
           {.to}{\bf to}2
           {.then}{\bf then}4
           {.else}{\bf else}4
           {.elseif}{\bf elseif}6
           {.end}{\bf end}3
           {.while}{\bf while}5
           {.end}{\bf end}3
           {.foreach}{\bf foreach}7
           {.to.}{\bf to}2
           {.downto}{\bf downto}6
           {.function}{\bf function}8
           {.case}{\bf case}4
           {.new}{\bf new}3
           {.return}{\bf return}6
           {.throw}{\bf throw}5
           {.for.}{\bf for}3
           {.repeat}{\bf repeat}6
           {.forever}{\bf forever}7
           {\\'A}{{\'A}}1
           {\\'E}{{\'E}}1
           {\\'I}{{\'I}}1
           {\\'O}{{\'O}}1
           {\\'U}{{\'U}}1
           {\\'a}{{\'a}}1
           {\\'e}{{\'e}}1
           {\\'i}{{\'i}}1
           {\\'o}{{\'o}}1
           {\\'u}{{\'u}}1
}

\newcommand{\TODO}[1]{\textcolor{red}{TODO: #1}}
\newcommand{\lleca}{\textsf{Lleca}\xspace}
\newcommand{\tok}[1]{\textcolor{red}{{\texttt{$\langle$#1$\rangle$}}}}
\newcommand{\nonterm}[1]{\textcolor{blue}{{\em{$\langle$#1$\rangle$}}}}
\newcommand{\produccion}[2]{
  \item[]
  \begin{tabularx}{\textwidth}{lrlr}
  #1 & $\xrightarrow{\hspace{.5cm}}$ & #2
  \end{tabularx}\\
}
\newcommand{\EMPTY}{$\epsilon$}
\newcommand{\ALT}{
  \\ & $\mid$ &
}
\newcommand{\first}{\texttt{FIRST}\xspace}
\newcommand{\follow}{\texttt{FOLLOW}\xspace}
\newcommand{\setunion}{\cup}
\newcommand{\ST}{\mid}
\newcommand{\set}[1]{\{#1\}}
\newcommand{\eps}{\epsilon}
\newcommand{\pal}{\alpha}
\newcommand{\paltwo}{\beta}
\renewcommand{\emptyset}{\varnothing}

\begin{document}
\begin{center}
\textsc{{\small Parseo y Generaci\'on de C\'odigo -- $2^{\text{do}}$ semestre 2017}} \medskip \\ 
\textsc{{\small Licenciatura en Inform\'atica con Orientaci\'on en Desarrollo de Software}} \medskip \\
\textsc{{\small Universidad Nacional de Quilmes}} \medskip \\
{\Large Trabajo pr\'actico 1} \medskip \\
{\large {\bf El parser gen\'erico \lleca}} \medskip \\
\bigskip
Fecha de entrega: 12 de octubre
\end{center}

\tableofcontents

\section{Introducci\'on}

Este TP consiste en implementar un lenguaje gen\'erico de an\'alisis
sint\'actico que llamaremos \lleca.
El parser \lleca recibe como entrada una gram\'atica que describe
un lenguaje. Por ejemplo, el archivo {\bf robot.ll} contiene la
especificaci\'on de un lenguaje para controlar un robot:\\
\begin{center}
\begin{tabular}{|p{.5\textwidth}|}
\hline
{\bf robot.ll}
\\
\hline
\begin{verbatim}
programa
| /* programa vacío */  => Fin
| comando programa      => Secuencia($1, $2)

comando
| "AVANZAR" NUM         => CmdAvanzar($2)
| "GIRAR" sentido       => CmdGirar($2)

sentido
| "IZQ"                 => Izquierda
| "DER"                 => Derecha
\end{verbatim}
\\
\hline
\end{tabular}
\end{center}
Cada producci\'on de la gram\'atica est\'a acompa\~nada de una acci\'on.
Las acciones se escriben despu\'es de una flecha ``\texttt{=>}''
e indican cu\'al
es el resultado que se debe asociar a dicha construcci\'on gramatical.
El resultado asociado a cada construcci\'on es siempre un \'arbol.
El parser \lleca recibe
adem\'as como entrada un archivo fuente, por ejemplo el siguiente:
\begin{center}
\begin{tabular}{|p{.5\textwidth}|}
\hline
{\bf esquina.input}
\\
\hline
\begin{verbatim}
AVANZAR 10 GIRAR DER AVANZAR 10
\end{verbatim}
\\
\hline
\end{tabular}
\end{center}
El parser \lleca analiza el archivo fuente con la gram\'atica indicada,
y produce un \'arbol como resultado.
Por ejemplo, si se invoca al parser con la gram\'atica
{\bf robot.ll} y el archivo fuente {\bf esquina.input},
se obtiene como resultado el siguiente \'arbol:
\begin{center}
\begin{tabular}{p{.5\textwidth}}
\begin{verbatim}
Secuencia( CmdAvanzar(10),
           Secuencia( CmdGirar(Derecha),
                      Secuencia( CmdAvanzar(10),
                                 Fin ) ) )
\end{verbatim}
\end{tabular}
\end{center}
En caso de que el archivo fuente no est\'e en el lenguaje generado
por la gram\'atica, \lleca debe emitir un mensaje de error.
En el contexto de este TP no es necesario que los mensajes de error sean
descriptivos\footnote{Pero cuanto m\'as descriptivos, mejor.}; alcanzar\'a con indicar que hay un error
de sintaxis.

\section{An\'alisis l\'exico}

En este apartado se describe c\'omo debe funcionar el analizador l\'exico
o tokenizador.
El mismo analizador l\'exico se utilizar\'a para segmentar tanto la
gram\'atica como el archivo fuente.
Los archivos que manipula \lleca constan de una secuencia de tokens:

\begin{itemize}
\item {\bf Identificadores}.
      Son simplemente nombres. En distintos lenguajes y contextos
      pueden representar diferentes cosas;
      por ejemplo, nombres de variables o funciones.
      Los identificadores son de la forma \texttt{[a-zA-Z\_][a-zA-Z0-9\_]*}
      es decir, constan de:
      \begin{itemize}
      \item Un primer caracter que debe ser alfab\'etico (\texttt{a, ..., z, A, ..., Z}) o un gui\'on bajo (\verb|_|).
      \item Seguido de una secuencia de $0$ o m\'as caracteres que pueden ser
            alfab\'eticos (\texttt{a, ..., z, A, ..., Z}), d\'igitos (\texttt{0, ..., 9}) o un gui\'on bajo (\verb|_|).
      \end{itemize}
      Ejemplos de identificadores:
      \begin{center}
      \verb|x    y    foo    BAR    _    CamelCase    f_o_o    foo42|
      \end{center}
      Idealmente los identificadores deber\'ian ser de longitud ilimitada, pero se acepta que la implementaci\'on
      se restrinja a identificadores de longitud entre 1 y 63.
\item {\bf N\'umeros.}
      Son constantes num\'ericas (enteros no negativos).
      Son de la forma \texttt{[0-9]+} es decir, constan de una secuencia
      de uno o m\'as d\'igitos.

      Ejemplos de constantes num\'ericas:
      \begin{center}
      \verb|0    1    2    001    42    123456789|
      \end{center}
      Idealmente las constantes num\'ericas deber\'ian ser de longitud ilimitada, pero se acepta que la implementaci\'on
      se restrinja a constantes num\'ericas entre $0$ y $2^{31} - 1$, que corresponde al m\'aximo entero positivo
      representable en un entero de 32 bits con signo.
\item {\bf Cadenas.}
      Constantes de cadena ({\em strings}).
      Empiezan con una comilla doble (\texttt{"}) y finalizan con una comilla doble (\texttt{"}).
      Todos los caracteres comprendidos entre la primera comilla y la segunda comilla se consideran
      el contenido del string.
      Adem\'as, se deben aceptar las siguientes secuencias de escape:
      \begin{itemize}
      \item \verb|\"| -- contrabarra (\verb|\|) seguida de una comilla doble (\verb|"|): representa una \'unica comilla doble (\verb|"|).
      \item \verb|\\| -- contrabarra (\verb|\|) seguida de contrabarra (\verb|\|): representa una \'unica contrabarra (\verb|\|).
      \end{itemize}
      Por ejemplo, la constante de cadena \verb|"Hola \"mundo\"."| representa un texto de longitud 13.

      Idealmente las cadenas deber\'ian ser de longitud ilimitada, pero se acepta que la implementaci\'on
      se restrinja a constantes num\'ericas de hasta 1023 caracteres.

      Ejemplos de constantes de cadena:
      \begin{center}
      \verb|""    "a"    "b"    "abc"    "\\"    "\"a\""|
      \end{center}

\item {\bf Literales: palabras clave y s\'imbolos reservados.}
      Llamamos {\em literales} a las palabras clave y s\'imbolos reservados del lenguaje.
      El conjunto de palabras clave y s\'imbolos reservados depende del lenguaje.
      Por ejemplo,
      en un lenguaje como Python la cadena \verb|and| es una palabra clave (que representa el ``y'' l\'ogico)
      y \verb|**| es un s\'imbolo reservado (que representa la potencia),
      mientras que en un lenguaje como C la palabra \verb|and| es simplemente un identificador (que podr\'ia
      ser el nombre de una variable o una funci\'on)
      y la cadena \verb|**| se tokeniza como una secuencia de dos s\'imbolos reservados
      (es decir, dos veces el literal \verb|*|).
      Es por eso que el tokenizador debe ser {\em param\'etrico}.

      Dicho de otro modo, cuando se crea un tokenizador, se deben proveer dos conjuntos de strings:
      \begin{itemize}
      \item Un conjunto $\mathcal{K}$ de {\bf palabras clave}.
      \item Un conjunto $\mathcal{S}$ de {\bf s\'imbolos reservados}.
      \end{itemize}

      Por ejemplo, si el conjunto de palabras clave es vac\'io,
      es decir $\mathcal{K} = \emptyset$,
      y el conjunto de s\'imbolos reservados es
      $\mathcal{S} = \{\verb|"+"|\}$,
      entonces el siguiente archivo de entrada:
      \begin{center}
      \texttt{if++x}
      \end{center}
      consta de cuatro tokens:
      \begin{enumerate}
      \item el identificador \verb|"if"|,
      \item el s\'imbolo reservado \verb|"+"|,
      \item el s\'imbolo reservado \verb|"+"|,
      \item el identificador \verb|"x"|.
      \end{enumerate}

      En cambio si el conjunto de palabras clave es $\mathcal{K} = \{\verb|"if"|\}$
      y el conjunto de s\'imbolos reservados es $\mathcal{S} = \{\verb|"++"|\}$,
      el mismo archivo de entrada consta de s\'olo tres tokens:
      \begin{enumerate}
      \item la {\bf palabra clave} \verb|"if"|,
      \item el s\'imbolo reservado \verb|"++"|,
      \item el identificador \verb|"x"|.
      \end{enumerate}

      El conjunto $\mathcal{K}$ de palabras clave debe contener \'unicamente palabras de la forma \verb|[a-zA-Z_][a-zA-Z0-9_]*|.
      Por ejemplo, las siguientes podr\'ian ser palabras clave en alg\'un lenguaje:
      \begin{center}
      \verb`if    then    else    while    return    BEGIN    END    _`
      \end{center}

      El conjunto $\mathcal{S}$ de s\'imbolos reservados debe contener \'unicamente s\'imbolos
      formados por cualquiera de los siguientes caracteres:
      \begin{center}
      \verb`(  )  [  ]  {  }  ,  ;  :  .  +  -  *  /  %  !  ?  $  @  #  |  &  =  <  >  ~  ^  \`
      \end{center}
      Por ejemplo, los siguientes podr\'ian ser s\'imbolos reservados en alg\'un lenguaje:
      \begin{center}
      \verb|+    :    ++    ->    >>    <*>    $$|
      \end{center}
      La \'unica excepci\'on es que un s\'imbolo reservado {\bf no} puede comenzar con \verb|/*|
      porque dicha combinaci\'on se utiliza para delimitar comentarios.

      Las palabras clave y s\'imbolos reservados se tratan de manera ligeramente distinta.
      Cuando el tokenizador reconoce una secuencia de caracteres de la forma \verb|[a-zA-Z_][a-zA-Z0-9_]*|
      que podr\'ian componer un identificador, verifica si se encuentra dentro del conjunto
      de palabras clave $\mathcal{K}$ o no. Por ejemplo, si $\mathcal{K} = \{\verb|"if"|\}$,
      el siguiente archivo de entrada:
      \begin{center}
        \verb|if x ifx|
      \end{center}
      consta de tres tokens:
      \begin{enumerate}
      \item la palabra clave \verb|"if"|.
      \item el identificador \verb|"x"|.
      \item el identificador \verb|"ifx"|.
      \end{enumerate}
      En cambio, en el caso de los s\'imbolos reservados, el tokenizador siempre consume el s\'imbolo m\'as largo que pueda consumir. Por ejemplo, si el lenguaje reconoce
      tanto el s\'imbolo \verb|+| como el s\'imbolo \verb|++|,
      es decir $\mathcal{S} = \{\verb|"+"|, \verb|"++"|\}$
      entonces el archivo de entrada que repite cinco veces el caracter \verb|+|:
      \begin{center}
        \verb|+++++|
      \end{center}
      consta de tres tokens:
      \begin{enumerate}
      \item el s\'imbolo reservado \verb|"++"|.
      \item el s\'imbolo reservado \verb|"++"|.
      \item el s\'imbolo reservado \verb|"+"|.
      \end{enumerate}
      Para conseguir este comportamiento se recomienda ordenar los
      s\'imbolos reservados de mayor a menor longitud.
      El tokenizador comprueba, uno por uno, si hay un prefijo
      de la entrada que coincida con alguno de los s\'imbolos
      reservados\footnote{La soluci\'on ideal es implementarlo con un
      aut\'omata finito o un trie,
      pero no es necesario en el contexto de este TP.}.
     
\item {\bf Espacios en blanco.}
      El tokenizador debe ignorar todos los espacios (\verb|' '|), tabs (\verb|'\t'|), retorno de carro (\verb|'\r'|) y fin de l\'inea (\verb|'\n'|).
\item {\bf Comentarios.}
      El tokenizador acepta comentarios comenzados con \verb|/*| es decir, una barra \verb|/| seguida de un asterisco \verb|*|
      y finalizan cuando se encuentra la primera ocurrencia de \verb|*/|
      es decir un asterisco \verb|*| seguido de una barra \verb|/|.
      Notar que el tokenizador {\bf no} reconoce comentarios anidados.
\item {\bf Notaci\'on para los s\'imbolos terminales.}
      Escribimos:
      \begin{itemize}
      \item \tok{identificador} para el s\'imbolo terminal que representa un identificador.
      \item \tok{n\'umero} para el s\'imbolo terminal que representa una constante num\'erica.
      \item \tok{cadena} para el s\'imbolo terminal que representa una constante de de cadena.
      \item \textcolor{red}{\texttt{"}\texttt{...}\texttt{"}} para los s\'imbolos terminales que representan literales.
      Por ejemplo \textcolor{red}{\texttt{"}\texttt{if}\texttt{"}}
      representa la palabra clave \verb|if|
      y \textcolor{red}{\texttt{"}\texttt{++}\texttt{"}} representa el s\'imbolo reservado \verb|++|.
      \end{itemize}
\end{itemize}

\subsection{Pseudoc\'odigo para el tokenizador}

\begin{lstlisting}
.function siguiente_token()
    Saltear espacios y comentarios.
    (*@$c$@*) := siguiente_caracter()
    .case (*@$c$@*) es un d\'igito:
         (*@$valor$@*) := leer una secuencia de d\'igitos de la entrada
         .return .new TokenNum\'erico(string_to_int((*@$valor$@*)))
    .case (*@$c$@*) es un caracter alfab\'etico o gui\'on bajo:
         (*@$valor$@*) := leer una secuencia de caracteres alfanum\'ericos de la entrada
         .if (*@$valor$@*) es una palabra clave
             .return .new TokenLiteral((*@$valor$@*))
         .else
             .return .new TokenIdentificador((*@$valor$@*))
         .end
    .case (*@$c$@*) es una comilla:
         (*@$valor$@*) := leer caracteres de la entrada hasta la siguiente comilla
         .return .new TokenCadena((*@$valor$@*))
    .else:
         .foreach s\'imbolo reservado (*@$sym$@*) en orden decreciente de longitud
             .if la entrada empieza con (*@$sym$@*)
                 .return .new TokenLiteral((*@$sym$@*))
             .end
         .end
         .throw (*@\textrm{``Error de sintaxis: caracter desconocido en la entrada''}@*)
    .end
.end
\end{lstlisting}

\section{Lectura de la gram\'atica}

\subsection{Descripci\'on informal del lenguaje \lleca}

Una vez que est\'e programado el tokenizador, se debe programar un
analizador sint\'actico para reconocer la gram\'atica
escrita en el lenguaje \lleca.

{\bf OJO:} no confundir este paso con el paso siguiente.
En este paso \'unicamente se lee un archivo escrito en lenguaje
\lleca y se construye una estructura de datos para representar
internamente la gram\'atica le\'ida.
Este paso {\bf NO} es el parser gen\'erico para analizar cualquier
archivo fuente sino un parser espec\'ifico para el lenguaje \lleca.

\begin{itemize}
\item
  Una {\bf gram\'atica} escrita en el lenguaje \lleca consta de una secuencia
  de $0$ o m\'as reglas. 
\item
  Cada {\bf regla} est\'a encabezada por un s\'imbolo no terminal
  seguido de una lista de $0$ o m\'as producciones.
\item
  Cada {\bf producci\'on}
  comienza con una barra vertical (\verb`"|"`),
  seguida de la expansi\'on (lado derecho de la producci\'on)
  y seguida de una acci\'on.
  La acci\'on est\'a dada por una flecha (\verb`"=>"`)
  y un t\'ermino. El t\'ermino representa el \'arbol
  que resulta de hacer el an\'alisis sint\'actico de dicha producci\'on.
\item
  Cada {\bf expansi\'on} es una lista de s\'imbolos.
  Los s\'imbolos pueden ser
  cadenas, identificadores, o cualquiera de las palabras
  clave \verb|"ID"|, \verb|"STRING"| o \verb|"NUM"|.
\item
  Un {\bf t\'ermino} es una expresi\'on recursiva que se construye
  de alguna de las siguientes maneras:
  \begin{itemize}
  \item La palabra clave \verb|"_"| (gui\'on bajo).
  \item Un identificador sin argumentos, por ejemplo, \texttt{Nil}.
  \item Un identificador acompa\~nado de una lista de
        argumentos encerrados entre par\'entesis y delimitados por comas,
        por ejemplo, \texttt{f(a, b, c)}
  \item Una constante de cadena.
  \item Una constante num\'erica.
  \item Una referencia a un par\'ametro, por ejemplo \verb|$3|.
  \item Una referencia a un par\'ametro seguida de un
        {\bf \'unico} argumento entre corchetes,
        por ejemplo \verb|$3[f(a, b, c)]|.
  \end{itemize}

  El siguiente es un ejemplo sencillo de gram\'atica:
  \begin{center}
  \begin{tabular}{|p{.8\textwidth}|}
 \hline
  {\bf alumnos.ll}
  \\
  \hline
  \begin{verbatim}
    base_de_alumnos
    | "begin" lista_alumnos "end" => $2

    lista_alumnos
    |                             => Nil
    | alumno ";" lista_alumnos    => Cons($1, $3)

    alumno
    | "#" NUM ":=" STRING         => Alumno("nombre", $4, "legajo", $2)
  \end{verbatim}
  \\
  \hline
  \end{tabular}
  \end{center}

\end{itemize}

\subsection{Gram\'atica del lenguaje \lleca}

Las palabras clave del lenguaje lleca son:
\begin{center}
    \verb|_    ID    STRING    NUM|
\end{center}
Los s\'imbolos reservados del lenguaje lleca son:
\begin{center}
    \verb`|    =>    $    (    )    ,    [    ]`
\end{center}
A continuaci\'on se dan todas las producciones de la gram\'atica
del lenguaje \lleca. El s\'imbolo inicial es \nonterm{gram\'atica}.
\begin{itemize}
\produccion{\nonterm{gram\'atica}}{
     \EMPTY
\ALT
     \nonterm{regla} \nonterm{gram\'atica}
}
\produccion{\nonterm{regla}}{
     \tok{identificador} \nonterm{producciones}
}
\produccion{\nonterm{producciones}}{
     \EMPTY
\ALT
     \nonterm{producci\'on} \nonterm{producciones}
}
\produccion{\nonterm{producci\'on}}{
     \textcolor{red}{\verb`"|"`} \nonterm{expansi\'on} \textcolor{red}{\verb|"=>"|} \nonterm{t\'ermino}
}
\produccion{\nonterm{expansi\'on}}{
     \EMPTY
\ALT \nonterm{s\'imbolo} \nonterm{expansi\'on}
}
\produccion{\nonterm{s\'imbolo}}{
     \textcolor{red}{\verb|"ID"|}
\ALT \textcolor{red}{\verb|"STRING"|}
\ALT \textcolor{red}{\verb|"NUM"|}
\ALT \tok{cadena}
\ALT \tok{identificador}
}
\produccion{\nonterm{t\'ermino}}{
     \textcolor{red}{\verb|"_"|}
\ALT \tok{identificador} \nonterm{argumentos}
\ALT \tok{cadena}
\ALT \tok{n\'umero}
\ALT \textcolor{red}{\verb|"$"|} \tok{n\'umero} \nonterm{sustituci\'on}
}
\produccion{\nonterm{argumentos}}{
     \EMPTY
\ALT \textcolor{red}{\verb|"("|} \nonterm{lista\_argumentos} \textcolor{red}{\verb|")"|}
}
\produccion{\nonterm{lista\_argumentos}}{
     \EMPTY
\ALT \nonterm{t\'ermino} \nonterm{lista\_argumentos\_cont}
}
\produccion{\nonterm{lista\_argumentos\_cont}}{
     \EMPTY
\ALT \textcolor{red}{\verb|","|} \nonterm{t\'ermino} \nonterm{lista\_argumentos\_cont}
}
\produccion{\nonterm{sustituci\'on}}{
     \EMPTY
\ALT \textcolor{red}{\verb|"["|} \nonterm{t\'ermino} \textcolor{red}{\verb|"]"|} 
}
\end{itemize}

\subsection{Analizador sint\'actico para el lenguaje \lleca}

Para analizar sint\'acticamente el lenguaje \lleca
pueden implementar un parser manualmente, usando la t\'ecnica de
an\'alisis sint\'actico por descenso recursivo.
Tambi\'en pueden usar un generador de parsers para su lenguaje favorito.
La gram\'atica que se provee arriba es LL(1).
Si utilizan un generador de parsers que utilice otra t\'ecnica de
an\'alisis sint\'actico como LALR(1),
es probable que tengan que adaptar ligeramente la gram\'atica
(por ejemplo, para eliminar la recursi\'on a derecha). 

\section{Procesamiento del archivo fuente}

En esta secci\'on se describe c\'omo analizar sint\'acticamente el archivo
fuente. El parser que deben implementar est\'a basado en la t\'ecnica de
an\'alisis sint\'actico LL(1).

\subsection{C\'alculo del conjunto de palabras clave y s\'imbolos reservados}

Una vez que se cuenta con una representaci\'on de la gram\'atica,
el primer paso es determinar cu\'ales son las palabras clave y
s\'imbolos reservados que usa dicha gram\'atica.
Para ello se deben recorrer los s\'imbolos que aparezcan en
las expansiones de todas las producciones.
Todos los s\'imbolos que sean cadenas representar\'an palabras clave o
s\'imbolos reservados.
\medskip

Por ejemplo, para la gram\'atica {\bf robot.ll}, tenemos:
\begin{itemize}
\item Palabras clave: $\mathcal{K} = \{\verb|"AVANZAR"|,\ \verb|"GIRAR"|,\ \verb|"IZQ"|,\ \verb|"DER"|\}$
\item S\'imbolos reservados: $\mathcal{S} = \emptyset$.
\end{itemize}
\medskip

Por otro lado, para la gram\'atica {\bf alumnos.ll}, tenemos:
\begin{itemize}
\item Palabras clave: $\mathcal{K} = \{\verb|"begin"|,\ \verb|"end"|\}$
\item S\'imbolos reservados: $\mathcal{S} = \{\verb|"#"|,\ \verb|":="|,\ \verb|";"|\}$.
\end{itemize}
\medskip

Observar que las cadenas que aparecen en las acciones (por ejemplo,
\verb|"nombre"| y \verb|"legajo"| en el caso de {\bf alumnos.ll})
no tienen por qu\'e ser palabras clave ni s\'imbolos reservados.

\subsection{Construcci\'on de la tabla de an\'alisis sint\'actico LL(1)}

El c\'alculo de la tabla LL(1) tiene tres pasos, que se describen en
las cuatro secciones siguientes.

\subsubsection{C\'alculo del conjunto \texttt{FIRST}}

El primer paso es calcular el conjunto de {\bf primeros} \texttt{FIRST}($X$) para cada s\'imbolo terminal o no terminal $X \in \Sigma \cup N$.
Recordar que el conjunto $\first(X)$ es un conjunto de s\'imbolos terminales y posiblemente el s\'imbolo $\eps$,
es decir $\first(X) \subseteq \Sigma \cup \set{\eps}$.
Un s\'imbolo terminal $a \in \Sigma$ est\'a en el conjunto $\texttt{FIRST}(X)$
cuando $X \Rightarrow^* a\beta$ para alguna cadena $\beta \in (N \cup \Sigma)^*$.
El s\'imbolo $\eps$ est\'a en el conjunto $\texttt{FIRST}(X)$
cuando $X$ es anulable, es decir $X \Rightarrow^* \eps$.
Se recuerda el pseudoc\'odigo para calcular el conjunto \texttt{FIRST}($X$) para
cada s\'imbolo $X \in \Sigma \cup N$:
\vspace{1cm}
\begin{lstlisting}
.entrada Una gram\'atica (*@$G = (N,\Sigma,P,S)$@*).
.salida  El conjunto (*@$\first(X)$@*) para cada s\'imbolo (*@$X \in N \setunion \Sigma$@*).
  (*@$\mathcal{F}$@*) := un diccionario (*@$\{X \mapsto \emptyset \ \ST\ X \in N \setunion \Sigma\}$@*)
  Poner (*@$\mathcal{F}$@*)[(*@$a$@*)] := (*@\set{$a$}@*) para cada s\'imbolo terminal (*@$a \in \Sigma$@*).
  .repeat hasta que no haya cambios
      .foreach producci\'on (*@$(A \to X_1 \hdots X_n) \in P$@*)
          .for. (*@$i$ = 1@*) to (*@$n$@*)
              .if (*@$X_1, \hdots, X_{i-1}$@*) son todos s\'imbolos anulables, es decir (*@$\eps \in \mathcal{F}[X_1] \cap \hdots \cap \mathcal{F}[X_{i-1}]$@*)
                  (*@$\mathcal{F}[A]$@*) := (*@$\mathcal{F}[A] \setunion \mathcal{F}[X_i]$@*)
              .end
          .end
          .if (*@$X_1, \hdots, X_{n}$@*) son todos s\'imbolos anulables, es decir (*@$\eps \in \mathcal{F}[X_1] \cap \hdots \cap \mathcal{F}[X_{n}]$@*)
              (*@$\mathcal{F}[A]$@*) := (*@$\mathcal{F}[A] \setunion \set{\eps}$@*)
          .end
      .end
  .end
  .return (*@$\mathcal{F}$@*)
\end{lstlisting}

El conjunto de primeros se puede generalizar para una cadena $\alpha \in (N \cup \Sigma)^*$.
Al igual que antes, $\first(\alpha) \subseteq \Sigma \cup \set{\eps}$,
de tal modo que un s\'imbolo terminal $a \in \Sigma$ est\'a en el conjunto $\texttt{FIRST}(\alpha)$
si y s\'olo si $\alpha \Rightarrow^* a\beta$ para alguna cadena $\beta \in (N \cup \Sigma)^*$.
El s\'imbolo $\eps$ est\'a en el conjunto $\texttt{FIRST}(\alpha)$
cuando la cadena $\alpha$ es anulable, es decir $\alpha \Rightarrow^* \eps$.
Se recuerda el pseudoc\'odigo para calcular el conjunto \texttt{FIRST}($\alpha$) para
una cadena arbitraria $\alpha \in (\Sigma \cup N)^*$:
\begin{lstlisting}
.entrada Una gram\'atica (*@$G = (N,\Sigma,P,S)$@*) y una cadena (*@$X_1 \hdots X_n \in (N \cup \Sigma)^\star$@*).
.salida  El conjunto (*@$\first(\pal)$@*).
  Computar el diccionario (*@$\mathcal{F}$@*) que a cada s\'imbolo (*@$X \in N \setunion \Sigma$@*) le asocia (*@$\first(X)$@*).
  (*@$\mathcal{R} := \emptyset$@*)
  .for. (*@$i$ = 1@*) to (*@$n$@*)
      .if (*@$X_1, \hdots, X_{i-1}$@*) son todos s\'imbolos anulables, es decir (*@$\eps \in \mathcal{F}[X_1] \cap \hdots \cap \mathcal{F}[X_{i-1}]$@*)
          (*@$\mathcal{R}$@*) := (*@$\mathcal{R} \setunion \mathcal{F}[X_i]$@*)
      .end
  .end
  .if (*@$X_1, \hdots, X_{n}$@*) son todos s\'imbolos anulables, es decir (*@$\eps \in \mathcal{F}[X_1] \cap \hdots \cap \mathcal{F}[X_{n}]$@*)
      (*@$\mathcal{R}$@*) := (*@$\mathcal{R} \setunion \set{\eps}$@*)
  .end
  .return (*@$\mathcal{R}$@*)
\end{lstlisting}


\subsubsection{C\'alculo del conjunto \texttt{FOLLOW}}

El segundo paso es calcular el conjunto de {\bf siguientes} \texttt{FOLLOW}($A$) para cada s\'imbolo no terminal $A \in N$.
Recordar que el conjunto $\follow(A)$ es un conjunto de s\'imbolos terminales y posiblemente el s\'imbolo $\$$,
es decir $\follow(X) \subseteq \Sigma \cup \set{\$}$.
Un s\'imbolo terminal $b \in \Sigma$ est\'a en el conjunto $\texttt{FOLLOW}(A)$
cuando $S \Rightarrow^* \gamma_1 A b \gamma_2$ para ciertas cadenas $\gamma_1, \gamma_2 \in (N \cup \Sigma)^*$,
donde $S$ es el s\'imbolo inicial.
El s\'imbolo $\$$ est\'a en el conjunto $\texttt{FOLLOW}(A)$
cuando $S \Rightarrow^* \gamma A$ para cierta cadena $\gamma \in (N \cup \Sigma)^*$.
Se recuerda el pseudoc\'odigo para calcular el conjunto \texttt{FOLLOW}($A$) para
cada s\'imbolo no terminal $A \in N$:

\begin{lstlisting}
.entrada Una gram\'atica (*@$G = (N,\Sigma,P,S)$@*).
.salida  El conjunto (*@$\follow(A)$@*) para cada s\'imbolo no terminal (*@$A \in N$@*).
  (*@$\mathcal{W}$@*) := un diccionario (*@$\{X \mapsto \emptyset \ \ST\ X \in N \setunion \Sigma\}$@*)
  (*@$\mathcal{W}[S]$@*) := (*@$\set{\$}$@*)
  .foreach producci\'on y s\'imbolo no terminal (*@$A \to \pal B \paltwo$@*)
      (*@$\mathcal{W}[B]$@*) := (*@$\mathcal{W}[B] \cup (\first(\paltwo) \setminus \set{\eps})$@*) 
  .end
  .while hay alg\'un cambio
      .foreach producci\'on y s\'imbolo no terminal (*@$A \to \pal B \paltwo$@*)
               tal que (*@$\paltwo$@*) es anulable, es decir (*@$\eps \in \first(\paltwo)$@*)
          (*@$\mathcal{W}[B]$@*) := (*@$\mathcal{W}[B] \cup \mathcal{W}[A]$@*) 
      .end
  .end
  .return (*@$\mathcal{W}$@*)
\end{lstlisting}

\subsubsection{C\'alculo de la tabla LL(1)}

La tabla de an\'alisis sint\'actico LL(1) indica, dado un s\'imbolo no terminal $A \in N$
y el siguiente s\'imbolo terminal $b \in \Sigma$, cu\'al es la producci\'on que debe
aplicar el algoritmo.

\begin{lstlisting}
.entrada Una gram\'atica (*@$G = (N,\Sigma,P,S)$@*).
.salida  La tabla de an\'alisis sint\'actico LL(1) para (*@$G$@*).
Poner (*@$\mathcal{T}[A,b] := \emptyset$@*) para todo (*@$(A,b) \in N \times \Sigma$@*).
.foreach producci\'on (*@$(A \to \pal) \in P$@*)
    .foreach s\'imbolo terminal (*@$x \in (\first(\pal) \setminus \set{\eps})$@*)
        (*@$\mathcal{T}[A,x]$@*) := (*@$\mathcal{T}[A,x] \cup \set{A \to \pal}$@*)
    .end
    .if (*@$\pal$@*) es anulable, es decir (*@$\eps \in \first(\pal)$@*)
        .foreach s\'imbolo (*@$x \in \follow(A)$@*)
            (*@$\mathcal{T}[A,x]$@*) := (*@$\mathcal{T}[A,x] \cup \set{A \to \pal}$@*)
        .end
    .end
.end
.return (*@$\mathcal{T}$@*)
\end{lstlisting}

Recordar que, si a alguna entrada le corresponden dos producciones distintas,
hay un conflicto en la tabla, de tal manera que la gram\'atica no es LL(1)
y no puede aplicarse este m\'etodo de an\'alisis sint\'actico.
En este caso \lleca deber\'ia reportar la presencia de un conflicto en la gram\'atica.

\subsection{An\'alisis sint\'actico}

Este paso requiere que se hayan calculado el conjunto de palabras clave $\mathcal{K}$ y s\'imbolos reservados $\mathcal{S}$
de la gram\'atica, y que se hayan calculado los conjuntos de s\'imbolos anulables, primeros
y siguientes.
El primer paso para analizar sint\'acticamente el archivo fuente es tokenizarlo con
los conjuntos $\mathcal{K}$ y $\mathcal{S}$ ya calculados.
A continuaci\'on se ejecuta el siguiente algoritmo recursivo \texttt{analizar}.
Recibe como par\'ametro un s\'imbolo terminal o no terminal $X \in N \cup \Sigma$.
De manera recursiva devuelve un \'arbol (m\'as precisamente un {\bf t\'ermino}),
ejecutando las acciones tal como las especifica la gram\'atica.

\begin{lstlisting}
.function analizar(X)
.entrada Un s\'imbolo terminal o no terminal (*@$X \in N \cup \Sigma$@*).
.salida  El AST que resulta de consumir el prefijo de la entrada que
         corresponde al s\'imbolo (*@$X$@*). Eleva un error de sintaxis en caso
         de que la entrada no est\'a en el lenguaje.
    .if (*@$X$@*) es un s\'imbolo terminal
        (*@$b$@*) := siguiente_token()           // consumir un token
        .if (*@$b \neq X$@*)
            .throw (*@\textrm{``Error de sintaxis: se esperaba leer $X$ pero se encontr\'o $b$.''}@*)
        .end
        .return un \'arbol que consta \'unicamente de la hoja (*@$X$@*), con su valor asociado.
    .else
        (*@$b$@*) := espiar_siguiente_token()    // mirar un token sin consumirlo
        .if (*@$\mathcal{T}[X,b] == \emptyset$@*)
          .throw (*@\textrm{``Error de sintaxis: se esperaba leer $X$ pero se encontr\'o $b$.''}@*)
        .end
        Sea (*@$(X \to Y_1 \hdots Y_n)$@*) la \'unica producci\'on en el lugar (*@$\mathcal{T}[X, b]$@*) de la tabla.
        (*@$args$ := []@*)
        .foreach (*@$i \in [1, ..., n]$@*)
          (*@$args$@*).agregar(analizar((*@$Y_i$@*)))
        .end
        .return un \'arbol que resulta de aplicar la acci\'on de la producci\'on (*@$X \to Y_1 \hdots Y_n$@*)
                con la lista de argumentos (*@$args$@*).
    .end
.end
\end{lstlisting}

\subsection{Acciones}

El proceso de an\'alisis sint\'actico construye recursivamente un {\bf t\'ermino}.
Los t\'erminos son \'arboles con la siguiente estructura (especificada usando un tipo
de datos de Haskell):
\begin{verbatim}
  data Termino = Agujero
               | Cadena String
               | Numero Int
               | Estructura String [Termino]
\end{verbatim}
En la notaci\'on del lenguaje \lleca, el siguiente es un t\'ermino posible:
\begin{verbatim}
  foo(bar("hola", 123), baz(_))
\end{verbatim}
que corresponde (en notaci\'on de Haskell) al siguiente valor:
\begin{verbatim}
  Estructura "foo" [
    Estructura "bar" [
      Cadena "hola",
      Numero 123
    ],
    Estructura "baz" [
      Agujero
    ]
  ]
\end{verbatim}

Se describe el significado de las acciones asociadas a cada producci\'on:

\begin{itemize}
\item {\bf Gui\'on bajo (\verb|_|):} construye un t\'ermino \texttt{Agujero}.
\item {\bf Cadena $s$:} construye un t\'ermino \texttt{Cadena $s$}.\\
      \verb|"hola"| $\rightsquigarrow$ \verb|Cadena "hola"|
\item {\bf N\'umero $n$:} construye un t\'ermino \texttt{Numero $n$}.\\
      \verb|123| $\rightsquigarrow$ \verb|Numero 123|
\item {\bf Identificador $id$ sin argumentos:} construye un t\'ermino \texttt{Estructura $id$ []}.\\
      \verb|foo| construye el \'arbol \verb|Estructura "foo" []|
\item {\bf Identificador $id$ con argumentos:} construye un t\'ermino \texttt{Estructura $id$ [...$argumentos$...]}.\\
      \verb|foo(bar, baz)| construye el \'arbol \verb|Estructura "foo" [Estructura "bar" [], Estructura "baz" []]|
\item {\bf Par\'ametro $\$n$:} devuelve el $n$-\'esimo argumento de la lista \texttt{argumentos} que resulta de la invocaci\'on 
      recursiva del algoritmo \texttt{analizar}.\\
      \verb|suma($1, $3)| $\rightsquigarrow$ \verb|Estructura "suma" [rec1, rec3]|\\
      donde \texttt{rec1} e \texttt{rec3} representan los valores de los s\'imbolos \texttt{Y1} e \texttt{Y3} en la producci\'on \verb|X -> Y1 ... Yn|.
\item {\bf Par\'ametro con sustituci\'on, $\$n[X]$ :}
      el $n$-\'esimo argumento de la lista \texttt{argumentos} es un t\'ermino
      que resulta de la invocaci\'on recursiva del algoritmo \texttt{analizar}.
      Cada vez que en dicho \'arbol haya una ocurrencia de \texttt{Agujero}
      se la debe reemplazar por el valor de \texttt{X}.

      Por ejemplo, con la siguiente gram\'atica:
      \begin{verbatim}
      cosa
      |          =>  _
      | NUM cosa =>  $2[suma(_, $1)])
      \end{verbatim}
      La entrada \verb|30| tiene como resultado:
      \begin{verbatim}
        suma(_, 30)
      \end{verbatim}
      La entrada \verb|20 30| tiene como resultado:
      \begin{verbatim}
        suma(suma(_, 20), 30)
      \end{verbatim}
      Y la entrada \verb|10 20 30| tiene como resultado:
      \begin{verbatim}
        suma(suma(suma(_, 10), 20), 30)
      \end{verbatim}
\end{itemize}

\section{Pautas de entrega}

Para entregar el TP se debe enviar el c\'odigo fuente por e-mail
a la casilla \texttt{foones@gmail.com} hasta las 23:59:59 del d\'ia
estipulado para la entrega, incluyendo \texttt{[TP lds-est-parse]} en
el asunto y el nombre de los integrantes del grupo en el cuerpo
del e-mail. No es necesario hacer un informe sobre el TP, pero se espera
que el c\'odigo sea razonablemente legible.
Se debe incluir un README
indicando las dependencias y el mecanismo de ejecuci\'on
recomendado para que el programa provea la funcionalidad pedida.
Se recomienda probar el programa con el conjunto de tests provistos.

\end{document}

